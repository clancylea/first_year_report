In order to process visual input from frame based imaging devices on a spiking 
neural network (SNN) a transformation is needed. The most common way is to 
simply encode using Poisson spiking with a rate that is proportional to pixel
intensity. This is an approximation to what the photoreceptor layer in the 
retina does, other cell layers react to changes in intensity and perform other 
computation before emitting actual spikes~\cite{webvision}. 

There has been multiple attempts at hardware based retina modelling, they've 
reported successful implementations with real-time performance, though require 
custom hardware~\cite{1465812,4145833,Vogelstein07amultichip}. 
So called \emph{silicon retinas} where first described by 
\citeauthor{carver-mead}~\cite{carver-mead}. Similar devices have been 
developed and reported, they are splendid real-time, low-powered, 
high-dynamic-range event-based cameras~\cite{aer-retina-bernabe, dvs-zurich}. 
This great hardware is in early stages of production, thus it may be too 
expensive for some researchers' budget or might not even be available for 
purchase.

Software based models have been reported by many authors with different 
results. One of the most accurate retinal models was developed by 
\citeauthor{virtual-retina}, whose results display
great levels of accuracy when compared to recorded data~\cite{virtual-retina}. 
\citeauthor{thorpe-rate-coding-theory} proposed a functional retinal model 
that uses 16 different ganglion cells to encode
images~\cite{thorpe-rate-coding-theory}. The model encodes images into 
\emph{Rank-Ordered} spikes, following the ideas of 
\citeauthor{field-sensory-coding} on sparse coding an redundancy
reduction~\cite{field-sensory-coding}. Although they obtain good results, their 
model lacked biological plausibility. Starting with this model, 
\citeauthor{basab-model} created a new one that takes into account the physical 
characteristics of the \emph{foveal pit} in the retina~\cite{basab-model}. We 
propose to implement this last model using parallel programming due to the 
nature of the problem. 

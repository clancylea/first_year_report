The literature review is about 60\%, though further reading might prove that 
this number might change. Our objective this year is to generate a 
video-to-spike train encoder. For this, the first approach was to use a
biologically plausible functional model \cite{basab-model} that results in
images being transformed into rank-ordered spikes 
\cite{thorpe-spike-rapid-processing}.

The retina is a thin layer of neural cells located in the eye, it is responsable
for the sensing, processing and transmissiting visual input\cite{webvision}. 
At its deepest layer, the retina has a millions of cells known as 
photoreceptors, they are in charge of transforming light into electrical 
signals. After this step there are, mainly, three layers of neurons that 
perform different computations such as lateral inhibition or on/off 
centre-off/on surround behaviour\cite{webvision-midget}. A small area 
at the centre of the retina has very few obstacles to obtain light and has high
resolution, this area is known as the \emph{foveal pit}.

This algorithm models the foveal pit, first we perform a two-dimensional  
\emph{convolution} of the current frame with four different \emph{kernels}
which represent four types of ganglion cells. After this step, a correction is 
needed due to redundancy of information in the convolved images. This last step
is carried in the retina via lateral inhibition prior to any ganglion cell 
activity. This project is near completion and we can encode video at 12 frames-per-second, though we believe some coding optimizations could lead to an increase in performance.\\

A second way of encoding is to simulate the early stages of the retina, which
sense changes in intensity on the photoreceptors. This is quite similar to what 
DVS do \cite{aer-retina-bernabe, dvs-zurich}, but with limited dynamic range 
and lower temporal resolution. The main advantage is that no specialized 
hardware is needed and the operation is so fast that any recent computer should
be able to do it. For this type of encoding procedure we hypothesize that
the bigger the change, the sooner a cell would spike and, thus, we can obtain
a spike timings given the difference of two video frames. This project is also
on its final stages, though more testing is required.


\subsection{Conclusions}
We obtained further knowledge about the anatomy of the eye from both a detailed 
and a functional approach. We also have come to appreciate the importance of 
inhibition circuits to enable high-efficiency, low-redundancy computing in the 
brain and how this brings robustness to neural structure. 

Real-time spike encoding, although with low temporal resolution, is achievable 
with a common GPU and the right combination of mathematics and 
engineering. Memory reads and writes in a GPU are extremely important, it is 
one of the biggest bottlenecks of the presented algorithms. If each convolution 
was to be performed by a different GPU, we expect to see much better 
performance, though testing is needed on this regard. We are planning on 
developing an FPGA based solution to this problem.

We propose timing mechanisms to emit spikes from a rank-ordered
source. Possible solutions for a faster mutual inhibition algorithm 
might be to do it in-line as we send spikes to neuromorphic hardware; 
or let the neural simulation deal with mutual inhibition.

\subsection{Future work}
We plan to explore learning on spiking neural networks using time-based codes or
rank-ordered ones. This is an area where work has been made but remains an
open problem.
After a learning mechanism is proposed, we will use this to design networks that
are able to robustly classify the MNIST dataset we previously encoded (section 
\ref{sec-project-progress}). Once some understanding of how features are
interpreted in the network is gained, we shall proceed to make use of this 
knowledge to derive vision algorithms. In some cases we will 
use multiple sensors, so we'll have to apply techniques like sensor fusion.


We obtained further knowledge about the anatomy of the eye from both a detailed 
and a functional approach. We also have come to appreciate the importance of 
inhibition circuits to enable high-efficiency, low-redundancy computing in the 
brain and how this brings robustness to neural structure. 

Real-time spike encoding, although with low temporal resolution, is achievable 
with common GPU and the right combination of mathematics and 
engineering. Memory reads and writes in a GPU are extremely important, it is 
one of the biggest bottlenecks of the presented algorithms.

We propose timing mechanisms to emit spikes from a rank-ordered
source. Possible solutions for a faster mutual inhibition algorithm 
might be to do it in-line as we send spikes to neuromorphic hardware; 
or let the neural simulation deal with mutual inhibition.

To reduce the power consumption/hardware requirements for mobile applications 
the best way to go might be change the resolution so not all image is 
perceived in high resolution.



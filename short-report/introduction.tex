Vision systems in biological entities are among the most complex sensory
inputs in nature. If we want to simulate them, it would require incredible
amounts of computing power and, traditionally, different algorithms to perform
each individual task. A parallel computation platform is the best way to go
while attempting to solve this problem, since neural structures in the brain 
compute in this way. 

In recent years neuromorphic (i.e. one that mimics the brain) hardware has
risen attention as a different way of computing. Platforms such as
\emph{SpiNNaker} emulate the brain's parallelism \cite{furber2013overview}; furthermore it does so while maintaining low power consumption. SpiNNaker 
can also give neural simulations the flexibility of software models
and biological real-time.

We hypothesize that a better understanding of vision in biology will lead to 
a unified computer vision framework. Using neural networks should translate in 
gaining an insight to the meaning of elements in a scene and, thus, a relation
between different images of the same scenario. This relationships should prove
to be a powerful tool to implement computer vision tasks such as registration, 
3D perception or optical flow, to name a few. Bio-inspired vision algorithms 
using SpiNNaker hardware could be used on robotics, security or transportation 
applications. A key element for any neural network approach is learning, as far
as we know, this remains an open problem for time or order based spike codes. 
The research on learning could lead into contributions on a theory of memory in 
the brain.

Converting conventional images or video into spike based representation is a
must-do step for further studies, in section \ref{sec-project-progress} we
report on the work done so far towards this goal. We use off-the-shelf hardware
to encode video, in particular we use Graphics Processing Units (GPU) due to 
their parallel architecture. In the final sections of this paper we present the plans we have to develop learning and classification algorithms.

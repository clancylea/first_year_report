In recent years neuromorphic (i.e. one that mimics the brain) hardware has
risen attention as a different way of computing. One key aspect is the high
parallelism found in the infrastructure of the brain. Platforms such as
\emph{SpiNNaker} \cite{furber2013overview} emulate such parallelism; furthermore
it does so while maintaining low power consumption. The \emph{SpiNNaker}
platform can also give neural simulations the flexibility of software models
and keep them running in biological real-time.

Converting conventional images or video into spike based representation is a
must-do step for further studies, in section \ref{sec-project-progress} we
report on the work done so far towards this goal. We use off-the-shelf hardware
to encode video, in particular we use Graphics Processing Units (GPU) due to 
their parallel architecture. 

In the final sections of this paper we present the plans we have to develop 
learning and classification algorithms. Furthermore this algorithms may lead to 
an implementation of vision tasks such as registration or optical flow.

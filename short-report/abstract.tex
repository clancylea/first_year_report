Vision systems in biological entities are among the most complex sensory
inputs in nature. If we want to simulate them, it would require incredible
amounts of computing power and, traditionally, different algorithms to perform
each individual task. A parallel computation platform is the best way to go
while attempting to solve this problem, since neural structures in the brain 
compute in this way. 

SpiNNaker is one of such platforms, a network of low-powered processing units,
each of which can simulate several neurons. Given that the SpiNNaker platform
resembles this natural neural structures, computer vision algorithms need to be
developed in a completely different manner.

The aim of this project is to develop algorithms in the realm of computer vision
but using a spiking neural networks approach. In particular we'll study 
time-based spike codes and how to process them. This algorithms should be 
able to cooperate and share their interpretation of the input data to gain a
more robust understanding of images.

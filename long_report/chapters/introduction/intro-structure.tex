As the inspiration for this work will be the properties of the brain, a description of the brain and its function can be found in Chapter~\ref{chp:brain} and we delve into the components of human vision in Chapter~\ref{chp:vision}.  

Neuromorphic hardware is a new trend in electronics and computer hardware design which takes inspiration from the brain, some examples of such platforms are explored in Chapter~\ref{chp:neuro-hw}.

An overview of the current state-of-the-art in 3D environment reconstruction is presented in Chapter~\ref{chp:reconstruction}.

Input for spiking neural networks has to be in \emph{spike trains}, which are a series of pulses emitted or received by a neuron in a given time slot. In order to use regular video sources they need be converted. There are few solutions which, mostly, require the use of custom hardware which is expensive may not be available to everyone. This year's work consisted on creating a software-based encoding procedure using parallel programming on off-the-shelf hardware; details of this can be found in Chapter~\ref{chp:img2spk}. 
%MAKE MY WORK SOUND BETTER
Conclusions and further work plans are presented in Chapter~\ref{chp:conclusions}.


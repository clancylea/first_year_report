For most animals, an important part of perception is done through visual input.

This kind of input has given humans the possibility of culture through reading and writing, cognitive development.

We might take vision for granted, but we are reminded of its importance when we hear an unusual noise in a dark room. 

An important aspect of vision is our ability to create mental maps of our current or, even past, locations.

The goal of the project is to create a 3D environment reconstruction.

There has been work on this field on ``classic'' computer vision, but for real-time they rely on high-performance power-hungry devices. Something that limits the actual utility of such systems for mobile applications. It would be a bit inconvenient to carry around a couple of car batteries in ones pocket.

We are able to do it with a highly-parallel 20-watts neural blob. There must be a more efficient way of doing this. A brief description of the brain and its function can be found in Chapter \ref{chp:brain}. We delve into the components of human vision in Chapter \ref{chp:vision}.

SpiNNaker provides a massively-parallel high-efficiency computing platform, inspired by the brain. It's an excellent choice for neuroscience research, particularly to study spiking neural networks. Its software stack has many ready-to-use neural models and development of new models can be performed in a straight forward manner. Chapter \ref{chp:neuro-hw} has a more detailed description of this and other neuromorphic hardware.\\


Input for spiking neural networks has to be in \emph{spike trains}, which are a series of spikes emitted by a neuron in a given time slot. In order to use video sources they need conversion. Few solutions which, mostly, require the use of custom hardware which is expensive and has low availability. We propose a parallel software based encoding.

This years work consisted in creating an input system for our spiking neural networks; details of this can be found in Chapter \ref{chp:img2spk}. 

While there are some examples of hardware based retinas, they are still expensive or they have limited availability. Implementing a retina model using consumer hardware is of great help for people that are unable to obtain a silicon retina.

Of special interest are mobile applications, if we can provide a low-power solution to a silicon retina emulator, we could enable millions of phones, tablets or computers to work as an input to neural computations (QUALCOMM CHIP, SPINNAKER) and keep the traditional camera functionality.

\section{Problem description}
\label{sec:intro:problem}
Environment reconstruction is an active field of research, particularly from the Simultaneous Localization and Mapping (SLAM) community\cite{Thrun2008_SLAM}. \textbf{SMALL DESCRIPTION OF SLAM!!!} 
%GPS unreliable because it's not precise, a couple of meters of difference could lead a car jumping on the curve or bumping another while trying to park.

Humans are able to do something similar with an efficient highly-parallel neural computing system that requires about 20-watts to function. How exactly this is done in the brain is still an open question. This research will provide a solution, inspired by state-of-the-art neuroscience, to the environment reconstruction problem using neuromorphic hardware.

Most research has cameras and a mixture of exotic depth sensors as inputs. One implication of this type of input is large quantities of information having to be processed, thus, needing high-performance and power-hungry devices to execute their algorithms. This is something that limits the actual utility of such systems for mobile applications. On the other hand, neuromorphic sensors have shown to reduce representations so that irrelevant information is not transmitted nor processed~\cite{aer-retina-bernabe,dvs-zurich}.

Another disadvantage to using ``classic'' computer vision approaches is that computational resources would need to be shared inefficiently (i.e. a processor would have to switch between a facial recognition algorithm to a depth-estimation one). Having a neuro inspired system means that the tasks are executed by the same network.

SpiNNaker provides a massively-parallel high-efficiency computing platform, inspired by the brain. It's an excellent choice for neuroscience research, particularly to study spiking neural networks. Its software stack has many ready-to-use neuron models and development can be performed in a straight forward manner~\cite{furber2014spinnaker}. 



\section{Objectives}
\label{sec:intro:objectives}
3D environment reconstruction is a very active field of research.
Advances in depth perception (KINECT) have made real-time simultaneous localization and mapping (SLAM) a possibility.
One way of achieving is to use high-performance GPUs and solve the problem using raw power.
Another is to use a mix of KINECT and RETINA, not fully neural???
We propose using an exclusively neural networks approach using SpiNNaker hardware.
\section{Plan}
\label{sec:intro:plan}
Steps:
\subsection{Image recognition}
\label{subsec:intro:plan:2D-recognition}
Time-based encoding, learning, classification, deep belief networks comparison, hierarchical structures

\subsection{3D object recognition}
\label{subsec:intro:plan:3D-recognition}
Correlation in space and time, spiking neural networks should make an excellent match for this.

\subsection{Depth perception}
\label{subsec:intro:plan:depth-perception}
Binocular, depth-from-defocus, other sensors? Optic flow to infer motion?

\subsection{Orientation and localization}
\label{subsec:intro:plan:localization}
Even more sensors? Make statistics/probabilistic models of past data?

\subsection{Reconstruction}
\label{subsec:intro:plan:reconstruction}
Get a top down approach? Interface 2 nets?

Vision is probably the greatest sensory input the brain has, it allows us to perceive a vastly diverse set of phenomenons. From enjoying a baby smile or checking if our meals are well cooked to avoiding traffic accidents or enjoying a film, it all happens to the amazing visual processing capabilities of the brain. Furthermore, this kind of input has given humans the possibility of culture through reading and writing, to the point that we are now starting to unravel the secret of how the brain works; in a sense, a brain trying to study itself.

What is known as vision can be thought of as a set of tasks that have been observed in humans and other animals, such as object detection and recognition, image segmentation or depth perception, to name a few. A complex but crucial aspect of vision is the ability to create mental maps of our current or past locations. This may have been developed as a survival strategy that allows animals to reach known food or water resources. Creating a system that mimics this extraordinary capacity is still an open research problem. Since it's been observed that our nervous system is able to recreate a mental representation of  environments, taking inspiration from nature is a good strategy to follow. The goal of this project is to create a bio-inspired 3D environment reconstruction system based on spiking neural networks. An additional benefit of this research is to acquire further knowledge of how the brain gains understanding and interprets the world through vision. Some applications to such a system are, mapping hazardous zones (e.g. nuclear power plants, war zones, active volcanoes), security (e.g. traffic camera analysis) or self or assisted guided vehicles (e.g. cars, drones, aeroplanes).

\section{Problem description}
\label{sec:intro:problem}
Environment reconstruction is an active field of research, particularly from the Simultaneous Localization and Mapping (SLAM) community\cite{Thrun2008_SLAM}. \textbf{SMALL DESCRIPTION OF SLAM!!!} 
%GPS unreliable because it's not precise, a couple of meters of difference could lead a car jumping on the curve or bumping another while trying to park.

Humans are able to do something similar with an efficient highly-parallel neural computing system that requires about 20-watts to function. How exactly this is done in the brain is still an open question. This research will provide a solution, inspired by state-of-the-art neuroscience, to the environment reconstruction problem using neuromorphic hardware.

Most research has cameras and a mixture of exotic depth sensors as inputs. One implication of this type of input is large quantities of information having to be processed, thus, needing high-performance and power-hungry devices to execute their algorithms. This is something that limits the actual utility of such systems for mobile applications. On the other hand, neuromorphic sensors have shown to reduce representations so that irrelevant information is not transmitted nor processed~\cite{aer-retina-bernabe,dvs-zurich}.

Another disadvantage to using ``classic'' computer vision approaches is that computational resources would need to be shared inefficiently (i.e. a processor would have to switch between a facial recognition algorithm to a depth-estimation one). Having a neuro inspired system means that the tasks are executed by the same network.

SpiNNaker provides a massively-parallel high-efficiency computing platform, inspired by the brain. It's an excellent choice for neuroscience research, particularly to study spiking neural networks. Its software stack has many ready-to-use neuron models and development can be performed in a straight forward manner~\cite{furber2014spinnaker}. 



\section{Objectives}
\label{sec:intro:objectives}
3D environment reconstruction is a very active field of research.
Advances in depth perception (KINECT) have made real-time simultaneous localization and mapping (SLAM) a possibility.
One way of achieving is to use high-performance GPUs and solve the problem using raw power.
Another is to use a mix of KINECT and RETINA, not fully neural???
We propose using an exclusively neural networks approach using SpiNNaker hardware.
%\section{Plan}
%\label{sec:intro:plan}
%Steps:
\subsection{Image recognition}
\label{subsec:intro:plan:2D-recognition}
Time-based encoding, learning, classification, deep belief networks comparison, hierarchical structures

\subsection{3D object recognition}
\label{subsec:intro:plan:3D-recognition}
Correlation in space and time, spiking neural networks should make an excellent match for this.

\subsection{Depth perception}
\label{subsec:intro:plan:depth-perception}
Binocular, depth-from-defocus, other sensors? Optic flow to infer motion?

\subsection{Orientation and localization}
\label{subsec:intro:plan:localization}
Even more sensors? Make statistics/probabilistic models of past data?

\subsection{Reconstruction}
\label{subsec:intro:plan:reconstruction}
Get a top down approach? Interface 2 nets?

\section{Report structure}
\label{sec:intro:structure}
As the inspiration for this work will be the properties of the brain, a description of the brain and its function can be found in Chapter~\ref{chp:brain} and we delve into the components of human vision in Chapter~\ref{chp:vision}.  

Neuromorphic hardware is a new trend in electronics and computer hardware design which takes inspiration from the brain, some examples of such platforms are explored in Chapter~\ref{chp:neuro-hw}.

An overview of the current state-of-the-art in 3D environment reconstruction is presented in Chapter~\ref{chp:reconstruction}.

Input for spiking neural networks has to be in \emph{spike trains}, which are a series of pulses emitted or received by a neuron in a given time slot. In order to use regular video sources they need be converted. There are few solutions which, mostly, require the use of custom hardware which is expensive may not be available to everyone. This year's work consisted on creating a software-based encoding procedure using parallel programming on off-the-shelf hardware; details of this can be found in Chapter~\ref{chp:img2spk}. 
%MAKE MY WORK SOUND BETTER
Conclusions and further work plans are presented in Chapter~\ref{chp:conclusions}.


Vision is one of the most important senses for animals; humans use it extensively for all kinds of tasks. Hunting, assessing danger, reading, driving, drawing, predicting rain from grey clouds, etc., these are all tasks that involve \emph{seeing}. 

There is a vast collection of knowledge about the components of vision (primates in particular), though a unification (or the answer to \emph{How do we see?}) has not yet been achieved.



Vision starts at the eye, which transforms electromagnetic radiation that assembles an image, into voltage pulses that our brain may interpret. This encoded images are sent to the posterior region of the brain through the optic nerves. The cortex then performs many computations that result in our ability to see.

\section{The eye and the retina}
\label{sec:vision:eye}
Our everyday experience might lead us to believe that the eyes are sensory organs developed completely separate from the brain but, in fact, the retina is an extension of the brain that performs spatio-temporal compression of a continuous flow of ``images'' of the world.

The eye is composed of many parts that resemble a camera (LENS, CAMERA OSCURA, FILM)

After light has been transformed into an electrical representation, the retina takes over and computes a representation of it.

Photoreceptors have the task of transforming light into an electrical signal. Colour is perceived by special type of receptors \emph{cones}. For low-light conditions and higher contrast sensitivity, we use \emph{rods}. Vertebrates have both rods and cones. Evolutionary adaptation has made eyes in different animals have special ratios of cones and rods. Reptiles and fish have more cones, most likely because they ``live'' on daytime for a lack of worm blood.

Many mammals have retinas with more rods than cones. For primates the retina has has two almost dual sensor zones. Most of the photosensitive area has more rods than cones; a tiny region called the \emph{foveal pit} has almost no rods, is densely packed with cones for high-resolution vision and is virtually blind when there is not enough light.

Horizontal cells average spatially (surround), input from photoreceptors; output to bipolar and to photoreceptors (adapt to different light conditions)

Bipolar cells, centre behaviour, input from horizontal and photoreceptors

IMAGE OF CENTRE SURROUND!!!!

First layers (photoreceptors, bipolar, horizontal cells) use analog signals, ganglion cells use spike trains.

Most authors agree that ganglion cells can be modelled by a \emph{Difference of Gaussians} due to its centre-surround behaviour.

Ganglion cells extend to the Lateral Geniculate Nucleus, where information is relayed and organized so that the cortex can interpret it. 

Organization makes left visual field sent to right hemisphere, right field to left hemisphere.



\section{The visual cortex}
\label{sec:vision:cortex}
The portion of the cortex that is involved with visual processing has been estimated to about 30\%.

It has been studied and areas have been labelled due to their function.

V1, V2, V...



\section{Conclusions}
\label{sec:vision:conclusions}
A complete model the visual cortex remains an open research question. Though many neural networks and computer vision models account for single visual functions, a unified solution has not been developed. Since environment reconstruction makes use of many of these functions, this research could lead to contributions to a general theory of vision. 

A hierarchical neural networks approach promises to have better results. The training of deep spiking neural networks is still in early stages of research, which brings another opportunity for contributions to neuroscience. Recurrent neural networks could provide the necessary framework to solve the training problem.

%\section{Eyes for computers}
%Traditionally cameras have been used as they work similar to the first stage of the eye.

Recently dynamic vision sensor

They work in a way that more resembles the retina, event based

Still need development, resolution is low, they do not perform multiple-scale
convolution, expensive

A mix of both can be a great tool for researchers and commercial applications see Chp \ref{chp:img2spk}
%\label{sec:vision:cameras}

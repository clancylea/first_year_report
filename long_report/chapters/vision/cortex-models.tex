Lowe's work inspired by neuro
Most of the models of visual functions involve a single visual area or a small number of them
Many of the models are feedforward 

bottom-up vs. top-down dichotomy

Spiking vs. non-spiking

first complex-simple cells \cite{hubel1962receptive}

Still, another operation common to many neurons in the visual cortex is the bell-shape tuning operation corresponding to a Gaussian-shaped response tuned to a specific, optimal pattern of activation from the presynaptic inputs (Poggio & Bizzi 2004), as induced, for instance, by a bar of a specific orientation for simple cells in V1 or by a face for neurons in specific patches of IT (Tsao et al. 2008). Other interesting models include ‘winner-take-all’ circuits (Yuille 1988; Rousselet et al. 2003). A close cousin of the ‘winner-take-all’ circuit is the softmax operation, which assumes the selection and transmission of the most active response among a set of neural inputs (Nowlan & Sejnowski 1995; Riesenhuber & Poggio 1999). It has been shown that such max-like operation successfully captures the response properties of a subset of the complex cells (Finn & Ferster 2007) in the primary visual cortex together with mechanisms based on the energy model by Adelson & Bergen (1985) described above. 

A number of models for early vision have been described (mostly in the eighties, following the work of Marr, Poggio, Ullman, Horn, Grimson, Richards, Winston, Ballard, Koch, Hildreth and others). These include models of edge detection (Marr 1979), spatio-temporal interpolation and approximation, computation of optical flow and direction selectivity (Ullman 1979; Marr 1981), computation of lightness and albedo, shape form contours, shape form texture, shape from shading, binocular stereo matching (Marr & Poggio 1976), structure from motion, structure from stereo, surface reconstruction (Grimson 1982) and filling-in (Ullman 1976), and computation of surface color (Barrow & Tenebaum 1981; Marr 1982; Hurlbert 1989). 

HMAX, Closely related hierarchical architectures include VisNet, hierarchical versions of normative approaches such as slow feature analysis, convolutional networks and other hierarchical architectures (Wersing & Konig, 2003, Ullman 2007, Masquelier & Thorpe 2007, Pinto et al. 2008). 

recognition, ventral stream

Hierarchical has been shown to provide geometric transformation invariance

Hierarchical neural networks for image interpretation

Hierarchical temporal memory 

feed-back 
prediction-verification recursions AI as “hypothesis-verification” HTM, Recurrent
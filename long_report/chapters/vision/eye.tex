Our everyday experience might lead us to believe that the eyes are sensory organs developed completely separate from the brain but, in fact, the retina is an extension of the brain that performs spatio-temporal compression of a continuous flow of ``images'' of the world.

The eye is composed of many parts that resemble a camera (LENS, CAMERA OSCURA, FILM)

After light has been transformed into an electrical representation, the retina takes over and computes a representation of it.

Photoreceptors have the task of transforming light into an electrical signal. Colour is perceived by special type of receptors \emph{cones}. For low-light conditions and higher contrast sensitivity, we use \emph{rods}. Vertebrates have both rods and cones. Evolutionary adaptation has made eyes in different animals have special ratios of cones and rods. Reptiles and fish have more cones, most likely because they ``live'' on daytime for a lack of worm blood.

Many mammals have retinas with more rods than cones. For primates the retina has has two almost dual sensor zones. Most of the photosensitive area has more rods than cones; a tiny region called the \emph{foveal pit} has almost no rods, is densely packed with cones for high-resolution vision and is virtually blind when there is not enough light.

Horizontal cells average spatially (surround), input from photoreceptors; output to bipolar and to photoreceptors (adapt to different light conditions)

Bipolar cells, centre behaviour, input from horizontal and photoreceptors

IMAGE OF CENTRE SURROUND!!!!

First layers (photoreceptors, bipolar, horizontal cells) use analog signals, ganglion cells use spike trains.

Most authors agree that ganglion cells can be modelled by a \emph{Difference of Gaussians} due to its centre-surround behaviour.

Ganglion cells extend to the Lateral Geniculate Nucleus, where information is relayed and organized so that the cortex can interpret it. 

Organization makes left visual field sent to right hemisphere, right field to left hemisphere.

Redundancy keeps things working even if some neurons/receptors die out. To avoid saturation of nerve fibres and over-representation lateral inhibition might play a big role. It's specially useful for spike-timing encoding, since sensors give a rate based output that needs to be re-encoded.
Our everyday experience might lead us to believe that the eyes are sensory organs developed completely separate from the brain but, in fact, the retina is an extension of the brain that performs spatio-temporal compression of a continuous flow of ``images'' of the world.

The eye is composed of many parts that resemble a camera (LENS, CAMERA OSCURA, FILM)

After light has been transformed into an electrical representation, the retina takes over and computes a representation of it.

First layers (bipolar, horizontal cells) use analog signals, ganglion cells use spike trains.

Ganglion cells extend to the Lateral Geniculate Nucleus, where information is relayed and organized so that the cortex can interpret it. 

Organization makes left visual field sent to right hemisphere, right field to left hemisphere.


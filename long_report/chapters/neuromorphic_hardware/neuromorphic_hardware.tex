Computer hardware has had an amazing increase on performance for decades, though for natively parallel problems these increases have not been enough. Special hardware emerged to solve specific tasks like video decoding or computer graphics, and general purpose processors adopted a multi-core design to alleviate performance needs. For neural-like computations, there has been an increase in custom hardware platforms which will be reviewed in this chapter.

\section{Classical computing}
classical computing
\section{Neuromorphic trends}
neuromorphic hardware trends
\section{SpiNNaker}
spinnaker info
%\section{Event-based model}
%event-based programming/infrastructure
\section{Conclusions}
conclusions neuro hardware
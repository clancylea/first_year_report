The term \emph{neuromorphic} (i.e. that resembles neural form) is attributed to \citeauthor{mead2012analog}, probably coined as they worked on the implementation of silicon retinas~\cite{carver-mead,mead2012analog}. The main advantages of hardware that lies on the neuromorphic range are, low power consumption, real-time functionality, scalability and fault tolerance. Further work has been done on neuromorphic sensors that have lead to silicon retinas, cochleas or visual motion sensors, to name a few~\cite{liu2010neuromorphic}. While sensory input is of utmost importance for every system, a platform to make use of these sensors for AI tasks is still an open research problem. 

Computer hardware has had an amazing increase on performance for decades, in fact it's been following an exponential trend known as \emph{Moore's Law} for decades. Natively parallel problems have had little benefit from these increases. Special hardware emerged to solve specific tasks like video decoding or computer graphics, and general purpose processors adopted a multi-core design to alleviate performance needs. For neural-like computations, there has been an increase in custom hardware platforms which will be reviewed in this chapter.

\section{Von-Neuman computing}
classical computing
\section{Neuromorphic trends}
neuromorphic hardware trends
\section{SpiNNaker}
spinnaker info
%\section{Event-based model}
%event-based programming/infrastructure
\section{Conclusions}
conclusions neuro hardware
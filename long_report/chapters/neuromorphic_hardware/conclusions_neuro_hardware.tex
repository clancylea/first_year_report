Neuromorphic hardware presents advantages in terms of energy efficiency and can help to develop ``intelligent'' platforms that could perform human-like tasks (e.g. image recognition) without the need of a computer that requires a power plant to operate. Among large scale hardware based neural platforms, TrueNorth has a nice balance of programmability and power consumption but it's lack of synaptic plasticity and restricted networking make it less desirable. 
SpiNNaker's power consumption could be diminished if the chip manufacturing process was reduced from its current $130 nm$. The network fabric on SpiNNaker brings unprecedented flexibility and the fact that neurons, synapses and plasticity are programmable, make it an excellent choice for neuroscience research.
Artificial intelligence (AI) is another field of computing which strains classical computers, so building systems that can achieve anything resembling intelligence requires very powerful computers. In 1997, IBM's Deep Blue computer won a game of chess to human master Garry Kasparov, relying on 480 specialized chess chips that performed massively parallel game moves searches~\cite{deep-blue-Campbell200257}. Although it was not the first attempt at hardware-based AI~\cite{indiveri2011frontiers}, it was probably the first time a general audience got to know that a computer could beat a human being on cognitive tasks. 

This great achievement may be dwarfed by the fact that humans can do much more than play chess and Deep Blue was nowhere near ready to do the laundry or pick up the kids from school. A general solution is still the holly grail of AI and, as research on the brain advances, scientists have taken inspiration from biology to develop new computing platforms. 
%Researchers have built custom hardware to solve AI problems as early as 1958~\cite{indiveri2011frontiers}. 
The term \emph{neuromorphic} (i.e. that resembles neural form) is attributed to \citeauthor{mead2012analog}, probably coined as he worked on the implementation of silicon retinas~\cite{carver-mead,mead2012analog}. The main advantages of hardware that lies on the neuromorphic range are, low power consumption, real-time functionality, scalability and fault tolerance. More work has been done on neuromorphic sensors that have lead to silicon retinas, cochleas or visual motion sensors, to name a few~\cite{liu2010neuromorphic}. While sensory input is of utmost importance for every system, a platform to make use of these sensors for AI tasks is still an open research problem. 

Hardware platforms that combine the knowledge acquired from neuroscience and artificial neural networks research have been recently developed. They may be classified based on the type of hardware/software combination used. Hardware neurons are analog circuits that behave like a mathematical model of a neuron; software neurons simulate the model using a general digital processor. Similar distinctions can be made for synapses, axonal and dendritic trees~\cite{misra2010artificial}. There are several projects implementing hardware neural platforms, the following are the most important ones~\cite{neuro-platforms-summary-7159144}.

The \emph{BrainScales} project is carried by multiple universities and research institutes - based on FACETS, exponential integrate and fire, short term depression and facilitation, stdp, programmable topology and parameters

\emph{Neurogrid} - custom model, hardware model, ram weights converted with Digital to Analog Converters (DACs), digital network for spikes

\emph{TrueNorth} - event driven, software neurons, binary programmable synapses, 
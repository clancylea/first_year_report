%\section{Conclusion}
Basic neural anatomy was revised and some mathematical models that reflect the behaviour of neurons were studied. A synaptic plasticity model was also briefly explained. Different types of spike encoding were analysed. Generations of neural networks were reviewed and the structure of models of the visual cortex were studied. 

This research will be done using hierarchical spiking neural networks, at least for the landmark identification and recognition task. Perhaps this same topology could prove itself useful for tracking or localization. Some combinations of simple and complex cells have had successful results, so they will form part of our solution. For the spike encoding, we believe that fast recognition applications require spike-time encoding.

Neuromorphic hardware options were considered and SpiNNaker is the best option for research purposes. It provides a balance between programming flexibility, energy efficiency and computing power. As far as sensors go, an encoder for odometer data would be developed and visual input would either come from our developed encoders or a DVS.

Solutions to the SLAM problem using only neural networks are scarce, to the best of our knowledge, only one paper has been reported to achieve such a feat. This opens an enormous possibility for research. Other work uses neural networks as an enhancement for known techniques.

\section{Publications}
The image and video encoding work during this year is part of a publication on the \emph{Benchmarks and Challenges for Neuromorphic Engineering} special issue of the Frontiers in Neuroscience journal.

\section{Further work}
Work on video conversion into spike trains is still ongoing. An FPGA-based solution is being studied, which could lead to a commercial product.

Studies of online learning with spike-time encoding are barely starting to appear, this is the foundation of our work so it will be researched extensively.

Hierarchical deep neural networks are the state-of-the-art in image recognition, understanding the relationship between them and a spiking equivalent is an interesting topic.



\section{Proposed thesis table of contents}
%\begin{multicols}{2}
  \begin{itemize}
      \item Abstract
      \item \textbf{Chapter 1}. Introduction.
      \begin{itemize}
        \item Neural networks.
        \item Spike codes in vision.
        \item Inhibition.
        \item Spatio-temporal patterns and learning.
        \item Research objectives.
      \end{itemize}
      \item \textbf{Chapter 2}. Background.
      \begin{itemize}
        \item SpiNNaker platform.
        \item Real-time artificial neural computations.
        \item Polychronization.
        \item Classification.
      \end{itemize}
      \item \textbf{Chapter 3}. Methodology.
      \begin{itemize}
      \item Model visual input using time-based spike codes.
      \item Hierarchical networks for robust classification.
      \item Feature identification.
      \item Sensor fusion and image registration.
      \end{itemize}
      \item \textbf{Chapter 4}. Results.
      \begin{itemize}
        \item Comparison with other methods.
        \item Discussion.
      \end{itemize}
      \item \textbf{Chapter 5}. Conclusions and Further Work.
      \begin{itemize}
        \item Conclusions.
        \item Future work.
        \item Publications.
      \end{itemize}
      \item \textbf{References}.
      
  \end{itemize}
%\end{multicols}

\section{Plan for second and third year}
%\input{./chapters/conclusions/plan}
The plan for next years is provided in the following Gantt chart.
\newpage
%\clearpage
%\newpage

\begin{figure}[t]
%  \begin{center}

    {\color{white}1}\hspace*{5cm}
    \begin{rotate}{270}
      \hspace*{-12cm}
      \begin{ganttchart}[x unit = 5mm,
                         y unit chart = 6mm,
                         vgrid={*1{themecolor, dotted}},
                         compress calendar, 
                         time slot format=isodate-yearmonth,
                         title/.append style={draw=none, fill=themecolordark},
                         title label font=\sffamily\bfseries\color{white},
                         title label node/.append style={below=-1.6ex},
                         title left shift=.05,
                         title right shift=-.05,
                         title height=1]{2014-09}{2017-08}
        \gantttitlecalendar{year, month} \\
         \ganttbar{Learn background concepts}{2014-09}{2015-02}\\
         \ganttbar{Implement retina models}{2014-12}{2015-04}\\
         \ganttbar{Familiarize with SpiNNaker}{2015-01}{2015-08}\\
         \ganttbar{Develop learning algorithms}{2015-07}{2016-04}\\
         \ganttbar{Develop classification networks}{2015-10}{2016-08}\\
         \ganttbar{Develop vision algorithms}{2016-08}{2017-04}\\
         \ganttbar{Theses writing}{2016-10}{2017-08}\\
      \end{ganttchart}

    \end{rotate}
    \label{fig:conc:project-plan}
%  \end{center}

\end{figure}

\newpage

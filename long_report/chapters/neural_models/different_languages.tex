In the previous sections, we discussed about the anatomy of the brain and neurons and how studies indicate that these neurons might communicate via electrochemical pulses known as spikes. Sensory input is transformed into a series of this action potentials, 

\subsection{Spike rate}
How many spikes where produced in a time slot. Not much information can be encoded this way. Easy to transfer previous neural net work. Not entirely biologically plausible, specially for high cognitive tasks.

\subsection{Spike timing}
The precise time a spike was emitted. Lots of information, but still difficult to use. Polychronization might be the answer to learning/training.

\subsection{Rank-order}
Only the order of spike events are important, not the particular timing. Might be more robust than spike-timing but can encode less information. Some problems on training as well.

Input from sensors is most likely rate-based, though processing time and energy consumption in the brain suggests a different one is used for further processing
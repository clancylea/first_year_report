Nervous systems in animals are different, from the simple ones found in insects to more complex ones in reptiles, birds and mammals. They are all mainly composed of a special kind of cell, the neuron, that excels at long distance communication. Most examples of nervous systems have a 

After millions of years of evolution, great apes, humans in particular,  have one of the most intricate nervous systems. The human brain acts as regulator of this system and performs high level cognitive tasks.

The human \textbf{brain} is an exquisite piece of evidence of energy-efficient biological computation. It's been subject of multiple studies and, yet, we are barely getting to know it. 

It consists of around $10^{12}$ individual cells which are interconnected through about $10^{15}$ synapses.


It can perform the most diverse activities, from bird spotting to mathematics to art. All of this with about 20 watts of energy spread across many small computational units called \emph{neurons}.



Most of this neurons are arranged in thin sheet of about 1100 cm$^2$ area and a 2 to 4 mm. thickness.

So far, several functional units in the brain have been identified; but the exact mechanisms of how they perform is still unknown.


One of the most consuming tasks is vision, about 30\% of the cerebral cortex is used in visual perception. 


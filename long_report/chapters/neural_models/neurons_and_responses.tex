The nervous system is composed of specialized cells called \emph{neurons}. Their area of expertise is long-range communication. While most cells in the body can ``talk'' to their neighbours, neurons have structures that allow them to communicate for up to XX cm (in the human case).

Neurons are composed of a soma (body), this part has similar components to other cells in the body. One of the specialized communication structures is called the \emph{axon}, through it the neuron outputs a signal. \emph{Dendrites} are at the other end of the information exchange and receive messages that other neurons sent through their axon, thus they can be seen as inputs for the cell.

Theories suggest they perform some kind of calculation, most of the time modelled as a threshold activation function

Some neurons use analog/continuous signals to transmit information, though they are mostly act as an interface to the exterior world.

Latest evidence suggests that the complex cognitive functions are performed using spikes, on-off responses, as a means for communication.

The place where axons and dendrites meet, is called the SSS, synapses

When one neuron's output elicits another neuron to spike, 


Environment reconstruction is an active field of research, particularly from the Simultaneous Localization and Mapping (SLAM) community\cite{Thrun2008_SLAM}. \textbf{SMALL DESCRIPTION OF SLAM!!!} GPS unreliable because it's not precise, a couple of meters of difference could lead a car jumping on the curve or bumping another while trying to park.

Humans are able to do something similar with an efficient highly-parallel neural computing system that requires about 20-watts to function. How exactly this is done in the brain is still an open question. This research will provide a solution, inspired by state-of-the-art neuroscience, to the environment reconstruction problem using neuromorphic hardware.

Most research has cameras and a mixture of exotic depth sensors as inputs. One implication of this type of input is large quantities of information having to be processed, thus, needing high-performance and power-hungry devices to execute their algorithms. This is something that limits the actual utility of such systems for mobile applications. On the other hand, neuromorphic sensors have shown to reduce representations so that irrelevant information is not transmitted nor processed.

Another disadvantage to using ``classic'' computer vision approaches is that computational resources would need to be shared inefficiently (i.e. a processor would have to switch between a facial recognition algorithm to a depth-estimation one). Having a neuro inspired system means that the tasks are executed by the same network.

SpiNNaker provides a massively-parallel high-efficiency computing platform, inspired by the brain. It's an excellent choice for neuroscience research, particularly to study spiking neural networks. Its software stack has many ready-to-use neuron models and development can be performed in a straight forward manner. 

